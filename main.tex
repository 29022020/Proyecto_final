\documentclass{article}
\usepackage[utf8]{inputenc}
\usepackage[spanish]{babel}
\usepackage{listings}
\usepackage{graphicx}
\graphicspath{ {images/} }
\usepackage{cite}

\begin{document}

\begin{titlepage}
    \begin{center}
        \vspace*{1cm}
            
        \Huge
        \textbf{Proyecto Final }
            
        \vspace{0.5cm}
        \LARGE
        Primeros Pasos.
            
        \vspace{1.5cm}
            
        \textbf{
        Juan Fernando Muñoz López.\\
        Carlos Daniel Lora Larios.
        }
            
        \vfill
            
        \vspace{0.8cm}
            
        \Large
        Despartamento de Ingeniería Electrónica y Telecomunicaciones\\
        Universidad de Antioquia\\
        Medellín\\
        Marzo de 2021
            
    \end{center}
\end{titlepage}

\tableofcontents
\newpage
\section{Sección introductoria: Ideación del proyecto final.}\label{intro}

Hasta este instante, el mundo de la programación a ambos nos ha atraído considerablemente, gracias a que nos ha abierto nuestra mente a una inmensidad de posibilidades y sobre todo a tener una mejor perspectiva del mundo de la computación, pero sobre todo en el mundo del software, puesto  que este es una parte de la computación a la que normalmente se le da la espalda por creer que es muy “compleja”; una vez nos encontramos en este punto con unas bases claras y sólidas, no tenemos otra actitud que vaya más allá de querer aprender mucho, y como no, de la mejorar manera en este curso, dando lo mejor de nosotros para poder ejecutar este proyecto a la altura de las exigencias y las circunstancias dadas por los profesores.


\section{Sección de contenido: Temática del proyecto.} \label{contenido}
Los videojuegos, a lo largo de la historia poco a poco logran tener cada vez un mayor nivel de influencia en la sociedad, gracias a que sin importar la edad; los tramas, historias, mundos, aventuras, gráficos, artes y sobre todo los estilos de juego logran atrapar a todos los públicos, llevándolos a sumergirse en aventuras excepcionales llenas de diversión y riego, en donde la habilidad es lo que marca la diferencia en quién es, el mejor. Logrando transversalizar a los juegos a otros sectores como la educación, consiguiendo que los educandos fortalezcan sus habilidades académicas mientras se divierten; en nuestro caso nos encargaremos de desarrollar uno desde cero.\\


Nosotros, ambos conocedores del mundo gamer, y con algunos años de experiencia jugando a videojuegos, decidimos a priori con lo poco que conocemos acerca de las posibilidades que nos ofrece el entorno de desarrollo Qt; optar por crear una especie de historia en la que prima la acción, la lucha y la aventura, porque además de ser nuestro estilo de juego favorito creemos que nos puede ofrecer un universo de posibilidades. Para ello, nuestra idea central, es crear una pequeña historia sobre la que se base nuestro juego para más allá de hacer que el usuario se divierta vaya pensando un poco en que se está contando con cada movimiento que hace.

\subsection{Ideación del proyecto.}

Como ya lo habíamos mencionado, nuestro objetivo básicamente es desarrollar un juego de acción, lucha y aventura. Para brindarle al jugador la posibilidad de controlar un personaje que pueda combatir contra monstruos para que adquiera experiencia y vaya avanzando de nivel logrando derrocar los jefes finales de cada nivel, nuestra intención es hacer que el personaje a controlar tenga sus propias características y habilidades para pelear y desenvolverse en el videojuego. Algunas cosas secundarias que tenemos pensadas hacen alusión a:\\

-Queremos que el juego cuente con efectos de sonido y algo de música para ambientar el desarrollo del videojuego.\\

- Que el personaje tenga varias funcionalidades como saltar, golpear, agacharse, usar habilidades especiales.\\

- Que el juego incluya algunos diálogos o npc (personaje no jugador) que contribuyan a la historia.\\

- Que haya indicadores de salud o de energía para los personajes.\\

- Que los monstruos tengan una apariencia espeluznante y una temática sombría y a que a medida que se avance sean más peligrosos.\\

- Niveles o mundos atractivos a la vista.\\




\section{Coclusión.} \label{Conclusión.}

Para finalizar nos gusatría comentar que objetivo más cercano es empezar a crear esa pequeña historia que sea aquella sobre la que se desarrolle el videojuego, para que paulatinamente vayamos adquiriendo las habilidades en programación, el manejo del lenguaje C++ y el entorno de desarrollo Qt, que son necesarias para enfrentarnos a este demandante proyecto, nuestro futuro videojuego.



\end{document}